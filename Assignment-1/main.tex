%hello world
\let\negmedspace\undefined
\let\negthickspace\undefined
\documentclass[journal,12pt,onecolumn]{IEEEtran}
\usepackage{cite}
\usepackage{amsmath,amssymb,amsfonts,amsthm}
\usepackage{algorithmic}
\usepackage{graphicx}
\usepackage{textcomp}
\usepackage{xcolor}
\usepackage{txfonts}
\usepackage{listings}
\usepackage{enumitem}
\usepackage{mathtools}
\usepackage{gensymb}
\usepackage{comment}
\usepackage[breaklinks=true]{hyperref}
\usepackage{tkz-euclide} 
\usepackage{listings}
\usepackage{gvv}                                        
%\def\inputGnumericTable{}                                 
\usepackage[latin1]{inputenc}                                
\usepackage{color}                                            
\usepackage{array}                                            
\usepackage{longtable}                                       
\usepackage{calc}                                             
\usepackage{multirow}                                         
\usepackage{hhline}                                           
\usepackage{ifthen}                                           
\usepackage{lscape}
\usepackage{tabularx}
\usepackage{array}
\usepackage{float}
\usepackage{multicol}


\newtheorem{theorem}{Theorem}[section]
\newtheorem{problem}{Problem}
\newtheorem{proposition}{Proposition}[section]
\newtheorem{lemma}{Lemma}[section]
\newtheorem{corollary}[theorem]{Corollary}
\newtheorem{example}{Example}[section]
\newtheorem{definition}[problem]{Definition}
\newcommand{\BEQA}{\begin{eqnarray}}
\newcommand{\EEQA}{\end{eqnarray}}
\newcommand{\define}{\stackrel{\triangle}{=}}
\theoremstyle{remark}
\newtheorem{rem}{Remark}

% Marks the beginning of the document
\begin{document}
\bibliographystyle{IEEEtran}
\vspace{3cm}

\title{ASSIGNMENT 1}
\author{EE24BTECH11031 - Jashwanth}
\maketitle

\bigskip

\begin{enumerate}
	\item The normal at a point $\vec{P}$ on the ellipse $x^2 +4y^2=16$ meets the $x$-axis at $\vec{Q}$. If $\vec{M}$ is the mid point of the line segment $\vec{PQ}$, then the locus of $\vec{M}$ interests the latusrectums of the given ellipse at the points
	
	\hfill (2009)
		\begin{enumerate}
				\begin{multicols}{2}
			\item $\brak{\pm\frac{3\sqrt{5}}{2},\pm\frac{2}{7}}$
			\item $\brak{\pm\frac{3\sqrt{5}}{2},\pm\sqrt{\frac{19}{4}}}$
				\columnbreak
			\item $\brak{\pm2\sqrt{3},\pm\frac{1}{7}}$
			\item $\brak{\pm2\sqrt{3},\pm\frac{4\sqrt{3}}{7}}$
				\end{multicols}
		\end{enumerate}
		
\item The locus of the orthocentre of the traingle formed by the lines
		
			$$\brak{1+p}x-py+p\brak{1+p}=0,$$
			$$\brak{1+q}x-qy+q\brak{1+q}=0,$$
		and $y=0$, where $p \neq q$, is
		\hfill(2009)
\begin{enumerate}
		\begin{multicols}{2}
	\item a hyperbola
	\item a parabola
		\columnbreak
	\item an ellipse
	\item a straight line
		\end{multicols}   
\end{enumerate}

\item Let $\vec{P}\brak{6,3}$ be a point on the hyperbola $\frac{x^2}{a^2}-\frac{y^2}{b^2}=1$. If the normal at the point $\vec{P}$ intersects the $x$-axis at $\brak{9,0}$, then the eccentricity of the hyperbola is 
	\hfill (2011)\\
		\begin{enumerate}
				\begin{multicols}{2}
			\item$\sqrt{\frac{5}{2}}$
			\item$\sqrt{\frac{3}{2}}$
				\columnbreak
			\item$\sqrt{2}$
			\item$\sqrt{3}$
				\end{multicols}
		\end{enumerate}

	\item Let $\brak{x,y}$ be any point on the parabola $y^2=4x$. Let $\vec{P}$ be the point that divides the line segment from $\brak{0,0}$ to $\brak{x,y}$ in the ratio $1:3$. Then the locus of $\vec{P}$ is  \hfill(2011)\\
		\begin{enumerate}
				\begin{multicols}{2}
			\item $x^2=y$
			\item $y^2=2x$
				\columnbreak
			\item $y^2=x$
			\item $x^2=2y$
				\end{multicols}
		\end{enumerate}

	\item The ellipse $E_{1}$:$\frac{x^2}{9}+\frac{y^2}{4}=1$ is inscribed in a rectangle $\vec{R}$ whose sides are parallel to the coordinate axes. Another ellipse $E_{2}$ passing through the point $\brak{0,4}$ circumscribes the rectangle $\vec{R}$.The eccentricity of the ellipse $E_{2}$ is \hfill(2012)\\

		\begin{enumerate}
				\begin{multicols}{2}
			\item $\frac{\sqrt{2}}{2}$
			\item $\frac{\sqrt{3}}{2}$
				\columnbreak
			\item $\frac{1}{2}$
			\item $\frac{3}{4}$
				\end{multicols}|
		\end{enumerate}

	\item The common tangents to the circie $x^2+y^2=2$ and the parabola $y^2=8x$ touch the circle at the points $\vec{P}$, $\vec{Q}$ and the parabola at the points $\vec{R}$, $\vec{S}$.Then the area of the quadrilateral $\vec{PQRS}$ is \hfill(JEE Adv. 2014)\\
		\begin{enumerate}
				\begin{multicols}{2}
			\item $3$
			\item $6$
				\columnbreak
			\item $9$
			\item $15$
				\end{multicols}
		\end{enumerate}
		
\end{enumerate}
\vspace{0.5cm}

		\textbf{D}.MCQs with One or More than One Correct
		
\begin{enumerate}
\item The number of the values of $c$ such that the straight line $y=4x+c$ touches the curves $(x^2/4)+y^2=1$ is \hfill(1998 - 2 Marks)\\
	\begin{enumerate}
			\begin{multicols}{2}
	\item $0$
	\item $1$
		\columnbreak
	\item $2$
	\item infinite
			\end{multicols}
	\end{enumerate}

\item If $\vec{P}=\brak{x,y}$, $\vec{F}_{1}=\brak{3,0}$, $\vec{F}_{2}=\brak{-3,0}$ and $16x^2+25y^2=400$, then $\vec{PF}_{1}+\vec{PF}_{2}$ equals \hfill(1998-2 Marks)\\
	\begin{enumerate}
			\begin{multicols}{2}
		\item $8$
		\item $6$
			\columnbreak
		\item $10$
		\item $12$
			\end{multicols}
	\end{enumerate}

\item On the ellipse $4x^2+9y^2=1$, the points at which the tangents are parallel to the line $8x=9y$ are \hfill(1999-3 Marks)\\
	\begin{enumerate}
			\begin{multicols}{2}
		\item $\brak{\frac{2}{5},\frac{1}{5}}$
		\item $\brak{-\frac{2}{5},\frac{1}{5}}$
			\columnbreak
		\item $\brak{-\frac{2}{5},-\frac{1}{5}}$
		\item $\brak{\frac{2}{5},-\frac{1}{5}}$
			\end{multicols}
	\end{enumerate}

\item The equations of the common tangents to the parabola $y=x^2$ and $y=-\brak{x-2}^2$ is/are \hfill(2006-5M,-1)\\
	\begin{enumerate}
			\begin{multicols}{2}
		\item $y=4\brak{x-1}$
		\item $y=0$
			\columnbreak
		\item $y=-4\brak{x-1}$
		\item $y=-30x-50$
			\end{multicols}
	\end{enumerate}

\item Let a hyperbola passes through the focus of the ellipse $\frac{x^2}{25}+\frac{y^2}{16}=1$. The transverse and conjugate axes of this hyperbola coincide with the major and minor axes of the given ellipse, also the product of eccentricities of given ellipse and hyperbola is $1$, then \hfill (2006-5M,-1)\\
	\begin{enumerate}
		\item the equation of hyperbola is $\frac{x^2}{9}-\frac{y^2}{16}=1$
		\item the equation of hyperbola is $\frac{x^2}{9}-\frac{y^2}{25}=1$
		\item focus of hyperbola is $\brak{5,0}$
		\item vertex of hyperbola is $\brak{5\sqrt{3},0}$
	\end{enumerate}

\item Let $\vec{P}\brak{x_{1}, y_{1}}$ and $\vec{Q}\brak{x_{2},y_{2}}$, $y_{1}<0,y_{2}<0$, be the end points of the latus rectum of the ellipse $x^2+4y^2=4$. The equations of parabolas with latus rectum $\vec{PQ}$ are \hfill(2008)\\
	\begin{enumerate}
			\begin{multicols}{2}
		\item $x^2+2\sqrt{3}y=3+\sqrt{3}$
		\item $x^2-2\sqrt{3}y=3+\sqrt{3}$
			\columnbreak
		\item $x^2+2\sqrt{3}y=3-\sqrt{3}$
		\item $x^2-2\sqrt{3}y=3-\sqrt{3}$
			\end{multicols}
	\end{enumerate}

\item In a traingle $\vec{ABC}$ with fixed base $\vec{BC}$, the vertex $\vec{A}$ moves such that
	$$\cos{B}+\cos{C}=4\sin^2{\frac{A}{2}}$$.
		If $a,b$ and $c$ denote the lengths of the sides of the traingle opposite to the angles $A,B$ and $C$, respectively, then \hfill(2009)\\
		\begin{enumerate}
			\item $b+c=4a$
			\item $b+c=2a$
			\item locus of the point $\vec{A}$ is an ellipse
			\item locus of the point $\vec{A}$ is a pair of straight lines
		\end{enumerate}

	\item The tangent $\vec{PT}$ and the normal $\vec{PN}$ to the parabola $y^2=4ax$ at a point $\vec{T}$ and $\vec{N}$, respectively. The locus of the centroid of the traingle $\vec{PTN}$ is a parabola whose \hfill(2009)\\
		\begin{enumerate}
				\begin{multicols}{2}
			\item vertex is $\brak{\frac{2a}{3},0}$
			\item directrix is $x=0$
				\columnbreak
			\item latus rectum is $\frac{2a}{3}$
			\item focus is $\brak{a,0}$
				\end{multicols}
		\end{enumerate}

\item An ellipse intersects the hyperbola $2x^2-2y^2=1$ orthogonally. The eccentricity of the ellipse is reciprocal of that of the hyperbola. If the axes of the ellipse are along the coordinate axes, then \hfill(2009)\\
		\begin{enumerate}
			\item equation of the ellipse is $x^2+2y^2=2$
			\item the foci of ellipse are $\brak{\pm1,0}$
			\item equation of the ellipse is $x^2+2y^2=4$
			\item the foci of ellipse are $\brak{\pm\sqrt{2},0}$
		\end{enumerate}











\end{enumerate}
\end{document} 
