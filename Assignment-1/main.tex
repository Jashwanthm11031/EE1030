%hello world
\let\negmedspace\undefined
\let\negthickspace\undefined
\documentclass[journal,12pt,onecolumn]{IEEEtran}
\usepackage{cite}
\usepackage{amsmath,amssymb,amsfonts,amsthm}
\usepackage{algorithmic}
\usepackage{graphicx}
\usepackage{textcomp}
\usepackage{xcolor}
\usepackage{txfonts}
\usepackage{listings}
\usepackage{enumitem}
\usepackage{mathtools}
\usepackage{gensymb}
\usepackage{comment}
\usepackage[breaklinks=true]{hyperref}
\usepackage{tkz-euclide} 
\usepackage{listings}
\usepackage{gvv}                                        
%\def\inputGnumericTable{}                                 
\usepackage[latin1]{inputenc}                                
\usepackage{color}                                            
\usepackage{array}                                            
\usepackage{longtable}                                       
\usepackage{calc}                                             
\usepackage{multirow}                                         
\usepackage{hhline}                                           
\usepackage{ifthen}                                           
\usepackage{lscape}
\usepackage{tabularx}
\usepackage{array}
\usepackage{float}


\newtheorem{theorem}{Theorem}[section]
\newtheorem{problem}{Problem}
\newtheorem{proposition}{Proposition}[section]
\newtheorem{lemma}{Lemma}[section]
\newtheorem{corollary}[theorem]{Corollary}
\newtheorem{example}{Example}[section]
\newtheorem{definition}[problem]{Definition}
\newcommand{\BEQA}{\begin{eqnarray}}
\newcommand{\EEQA}{\end{eqnarray}}
\newcommand{\define}{\stackrel{\triangle}{=}}
\theoremstyle{remark}
\newtheorem{rem}{Remark}

% Marks the beginning of the document
\begin{document}
\bibliographystyle{IEEEtran}
\vspace{3cm}

\title{ASSIGNMENT 1}
\author{EE24BTECH11031 - Jashwanth}
\maketitle
\newpage
\bigskip

\begin{enumerate}
\item The normal at a point Pon the ellipse  $x^2 +4y^2=16$ meets the x-axis at Q. If M is the mid point of the line segment PQ, then the locus of M interests the latusrectums of the given ellipse at the points
	
	\hfill (2009)
		\begin{enumerate}
			\item $\brak{\pm\frac{3\sqrt{5}}{2},\pm\frac{2}{7}}$
			\item $\brak{\pm\frac{3\sqrt{5}}{2},\pm\sqrt{\frac{19}{4}}}$
			\item $\brak{\pm2\sqrt{3},\pm\frac{1}{7}}$
			\item $\brak{\pm2\sqrt{3},\pm\frac{4\sqrt{3}}{7}}$	
		\end{enumerate}
		
	\item The locus of the orthocentre of the traingle formed by the lines\\
		\begin{center}
		$(1+p)x-py+p(1+p)=0$,\\
		$(1+q)x-qy+q(1+q)=0$,\\
		\end{center}
		and y=0, where $p \neq q$, is
		\hfill(2009)
\begin{enumerate}
	\item a hyperbola
	\item a parabola
	\item an ellipse
	\item a straight line
\end{enumerate}

\item Let P(6,3) be a point on the hyperbola $\frac{x^2}{a^2}-\frac{y^2}{b^2}=1$. If the normal at the point P intersects the x-axis at (9,0),then the eccentricity of the hyperbola is
	\hfill (2011)\\
		\begin{enumerate}
			\item$\sqrt{\frac{5}{2}}$
			\item$\sqrt{\frac{3}{2}}$
			\item$\sqrt{2}$
			\item$\sqrt{3}$
		\end{enumerate}

	\item Let (x,y) be any point on the parabola $y^2=4x$. Let P be the point that divides the line segment from (0,0) to (x,y) in the ratio 1:3. Then the locus of P is  \hfill(2011)\\
		\begin{enumerate}
			\item $x^2=y$
			\item $y^2=2x$
			\item $y^2=x$
			\item $x^2=2y$
		\end{enumerate}

	\item The ellipse $E_{1}$:$\frac{x^2}{9}+\frac{y^2}{4}=1$ is inscribed in a rectangle R whose sides are parallel to the coordinate axes. Another ellipse $E_{2}$ passing through the point (0,4) circumscribes the rectangle R.The eccentricity of the ellipse $E_{2}$ is \hfill(2012)\\

		\begin{enumerate}
			\item $\frac{\sqrt{2}}{2}$
			\item $\frac{\sqrt{3}}{2}$
			\item $\frac{1}{2}$
			\item $\frac{3}{4}$
		\end{enumerate}

	\item The common tangents to the circie $x^2+y^2=2$ and the parabola $y^2=8x$ touch the circle at the points P,Q and the parabola at the points R,S.Then the area of the quadrilateral PQRS is \hfill(JEE Adv. 2014)\\
		\begin{enumerate}
			\item 3
			\item 6
			\item 9
			\item 15
		\end{enumerate}
		\vspace{0.5cm}
\end{enumerate}


		\textbf{D}.MCQs with One or More than One Correct
		\vspace{0.5cm}
\begin{enumerate}
\item The number of the values of c such that the straight line $y=4x+c$ touches the curves ($x^2$/4)+$y^2$=1 is \hfill(1998 - 2 Marks)\\
	\begin{enumerate}
	\item 0
	\item 1
	\item 2
	\item infinite
	\end{enumerate}i

\item If P=(x,y),$F_{1}$=(3,0),$F_{2}$=(-3,0) and $16x^2+25y^2=400$,then P$F_{1}$+P$F_{2}$ equals \hfill(1998-2 Marks)\\
	\begin{enumerate}
		\item 8
		\item 6
		\item 10
		\item 12
	\end{enumerate}

\item On the ellipse $4x^2+9y^2=1$ , the points at which the tangents are parallel to the line $8x=9y$ are \hfill(1999-3 Marks)\\
	\begin{enumerate}
		\item $\brak{\frac{2}{5},\frac{1}{5}}$
		\item $\brak{-\frac{2}{5},\frac{1}{5}}$
		\item $\brak{-\frac{2}{5},-\frac{1}{5}}$
		\item $\brak{\frac{2}{5},-\frac{1}{5}}$
	\end{enumerate}

\item The equations of the common tangents to the parabola $y=x^2$ and $y=-(x-2)^2$ is/are \hfill(2006-5M,-1)\\
	\begin{enumerate}
		\item $y=4(x-1)$
		\item $y=0$
		\item $y=-4(x-1)$
		\item $y=-30x-50$
	\end{enumerate}

\item Let a hyperbola passes through the focus of the ellipse \textbf{$\frac{x^2}{25}+\frac{y^2}{16}=1$}. The transverse and conjugate axes of this hyperbola coincide with the major and minor axes of the given ellipse, also the product of eccentricities of given ellipse and hyperbola is 1 , then \hfill (2006-5M,-1)\\
	\begin{enumerate}
		\item the equation of hyperbola is $\frac{x^2}{9}-\frac{y^2}{16}=1$
		\item the equation of hyperbola is $\frac{x^2}{9}-\frac{y^2}{25}=1$
		\item focus of hyperbola is (5,0)
		\item vertex of hyperbola is (5$\sqrt{3}$,0)
	\end{enumerate}

\item Let P($x_{1},y_{1}$) and Q($x_{2},y_{2}$), $y_{1}<0,y_{2}<0$, be the end points of the latus rectumof the ellipse $x^2+4y^2=4$. The equations of parabolas with latus rectum PQ are \hfill(2008)\\
	\begin{enumerate}
		\item $x^2+2\sqrt{3}y=3+\sqrt{3}$
		\item $x^2-2\sqrt{3}y=3+\sqrt{3}$
		\item $x^2+2\sqrt{3}y=3-\sqrt{3}$
		\item $x^2-2\sqrt{3}y=3-\sqrt{3}$
	\end{enumerate}

\item In a traingle ABC with fixed base BC, the vertex A moves such that\\
	\hspace{1.5cm}$cosB+cosC=4sin^2\frac{A}{2}$.\\
		If a,b and c denote the lengths of the sides of the traingle opposite to the angles A,B and C, respectively, then \hfill(2009)\\
		\begin{enumerate}
			\item $b+c=4a$
			\item $b+c=2a$
			\item locus of the point A is an ellipse
			\item locus of the point A is a pair of straight lines
		\end{enumerate}

	\item The tangent PT and the normal PN to the parabola $y^2=4ax$ at a point T and N, respectively. The locus of the centroid of the traingle PTN is a parabola whose \hfill(2009)\\
		\begin{enumerate}
			\item vertex is $\brak{\frac{2a}{3},0}$
			\item directrix is x=0
			\item latus rectum is $\frac{2a}{3}$
			\item focus is (a,0)
		\end{enumerate}

	\item An ellipse intersects the hyperbola $2x^2-2y^2=1$ orthogonally. The eccentricity of the ellipse is reciprocal of that of the hyperbola. If the axes of the ellipse are along the coordinate axes, then \hfill (2009)\\
		\begin{enumerate}
			\item equation of the ellipse is $x^2+2y^2=2$
			\item the foci of ellipse are $(\pm1,0)$
			\item equation of the ellipse is $x^2+2y^2=4$
			\item the foci of ellipse are $(\pm\sqrt{2},0)$
		\end{enumerate}











\end{enumerate}
\end{document} 
